\section{Lecture 1 --- 10th August 2022 --- Review of things done in the previous semester\ldots}



\subsection{Definitions and Some Results}
\begin{definition}[algebra]
    Let $\Omega$ be nonempty set. An algebra $\scrF$ is a collection of subsets of $\Omega$ satisfying the following properties:
    \begin{enumerate}
	\item $\Omega \in \scrF$,
	\item $A\in \scrF \implies A^c \in \scrF$ and
	\item $\scrF$ is closed under finite unions.
    \end{enumerate}
    \label{def:algebra}
\end{definition}

\begin{remark}
    It immediately follows from the definition that an algebra of sets is closed under taking finite intersections. Take compliment of finite intersections and make use of De Morgan's Theorem.
\end{remark}

\begin{definition}[$\sigma$-algebra]
    Let $\Omega$ be nonempty set. A $\sigma$-algebra $\scrF$ is a collection of subsets of $\Omega$ satisfying the following properties:
    \begin{enumerate}
	\item $\Omega \in \scrF$,
	\item $A\in \scrF \implies A^c \in \scrF$ and
	\item $\scrF$ is closed under countable unions.
    \end{enumerate}
    \label{def:sigma-algebra}
\end{definition}

\begin{remark}
    Similar to what we saw in an algebra of sets, the $\sigma$-algebra of sets is also closed under countable intersection. The proof is similar to that of the same with algebra of sets.
\end{remark}

\begin{fact}
    Let $\Omega$ be a set, $\scrF \subseteq \calP (\Omega )$. $\scrF$ is an $\sigma$-algebra iff $\scrF$ is an algebra that is continuous from below, that is, if $\left\{ A_n \right\}_{n\in \N} \subseteq \scrF$ and $A_n \subset A_{n+1}$ for all $n\in \N$ then $\bigcup _n A_n \in \scrF$.
\end{fact}

\begin{definition}[$\sigma$-algebra generated by a subset of power set]
    Let $\Omega$ be a nonempty set. Given an nonempty collection $\calC$ of subsets of $\Omega$, the $\sigma$-algebra generated by $\calC$, $\sigma (\calC)$ is defined to be the intersection of all $\sigma$-algebra containing $\calC$. Notationally,
    $$\sigma \left( \calC \right) = \bigcap \left\{ \sigma-\text{algebra that contains  }\calC \right\}$$
    \label{def:generated-sigma-algebra}
\end{definition}

$\sigma(\scrC)$ is the smalled $\sigma$-algebra containing $\scrC$

\begin{definition}[Borel $\sigma$-algebra]
    If $\Omega$ is a topological space then the Borel $\sigma$-algebra is the smallest $\sigma$-algebra containing the open sets of $\Omega$. $\ie$ by definition Borel $\sigma$-algebra is the $\sigma$-algebra generated by open sets in $\Omega$.
    \label{def:borel-sigma-algebra}
\end{definition}


\begin{fact}
    If \ $\Omega = \R ^n$ the Borel $\sigma$-algebra is generated by
    \begin{itemize}
	\item $\left\{ \left( a_1 , b_1 \right) \times \left( a_2 , b_2 \right) \times\cdots \times \left( a_n , b_n \right) \, \mid \, -\infty \le a_i < b_i \le +\infty  \right\} $
	\item $\left\{ \left( -\infty , a_1 \right) \times \left( -\infty , a_2 \right) \times \cdots \left( -\infty , a_n \right) \, \mid \, a_1 ,a_2, \ldots , a_n \in \R \right\}$
	\item $\left\{ \left( a_1 , b_1 \right)\times \left( a_2 , b_2 \right) \times \cdots \times \left( a_n , b_n \right) \, \mid \, a_i , b_i \in \Q \right\}$
    \end{itemize}
\end{fact}

\begin{definition}[$\pi$-system, $\lambda$-system]
    A collection $\calC$ of subsets of $\Omega$ is called a $\pi$-system if $\calC$ is closed under finite intersections.

    A collection $\calL$ of subsets of $\Omega$ is called a $\lambda$-system if the following hold:
    \begin{itemize}
	    \item $\Omega \in \calL$,
	    \item $A,B\in \calL$ and $A\subset B$ implies $B\setminus A \in \calL$
	    \item if $\left\{ A_n \right\}_{n \in \N} \subset \calL$ and $A_n \subset A_{n+1}$ for all $n\in \N$ then $\bigcup_{n} A_n \in \calL$
    \end{itemize}
    \label{def:pi-system}
    \label{def:lambda-system}
\end{definition}

\begin{definition}
    Let $\calC$ be a collection of nonempty subsets of a nonempty set $\Omega$. The $\lambda$-system generated by $\calC$, denoted as $\lambda (\calC )$ is the intersection of all $\lambda$-systems containing $\calC$.
    \label{def:generated-lambda-system}
\end{definition}



\subsection{Dynkin's pi-lambda theorem; Measures and their properties}

\begin{theorem}[Dynkin $\pi - \lambda$ theorem]
    If $\calC$ is a $\pi$-system of a nonempty set $\Omega$ then $\lambda (\calC) = \sigma (\calC)$. Equivalently, if $\calL$ is a $\lambda$-system that contains $\calC$ then $\lambda (\calC) \subset \calL$.
    \label{thm:dynkin}
\end{theorem}

\begin{definition}
    Let $\calF$ be a $\sigma$-algebra of subsets of $\Omega$. A extended real valued function $\mu$ on $\calF$ is called a \textit{measure} if the following hold:
    \begin{enumerate}
	\item $\mu \left(A\right) \ge 0$ for all $A\in \calF$
	\item $\mu \left( \emptyset \right) = 0$
	\item If $\left\{ A_n \right\}_{n \in \N} \subset \calF$ such that $\bigcup A_n \in \calF$ and $A_n \cap A_m =\emptyset$ for all $m\ne n$ then $\mu \left( \bigcup_{i} A_i \right) =\sum_{i} \mu (A_i)$
    \end{enumerate}
    \label{def:measure}
\end{definition}

\begin{example}[Some examples of measures]
    \begin{enumerate}

	\item Let $\Omega \ne \emptyset$, $\calF = \calP (\Omega)$. We define $\mu$ on $\calF$ by $\mu (A)$ is the number of elements of $A$ if $A$ is finite and $\mu = +\infty$ if $A$ contains infinitely many elements. Then $\mu$ is a measure on $\calF$ called the counting measure on $\calF$.

	\item Let $\Omega =[0,1]$ and $\calF = \calB \left( \Omega \right)$. Let $\left\{ p_n \right\}$ be a sequence of numbers in $[0,1]$ such that $\sum p_i =1$. Define $\mu (A) = \sum_{i\in\N} p_i \delta _{p_i} \left( A \right)$. Then $\mu $ is a measure on $\calF$.

	\item Let $F$ be a non-decreasing right-continuous function on $\R$. Define $\mu_F$ to be Lebesgue-Stieljes measure induced by $F$. Recall that $\mu_F \left( (a,b] \right) = F(b)-F(a)$. Then $\mu _F$ is an example of $\sigma$-finite Radon measure on the Borel $\sigma$-algebra on $\R$.

    \end{enumerate}
\end{example}

\begin{theorem}
    Let $\calF$ be a \ $\sigma$-algebra on a nonempty set $\Omega$. Let $\mu : \calF \to  [0, \infty]$ be a function. $\mu$ is a measure on $\calF$ iff 
	\begin{enumerate}
	    \item$\mu$ is finitely additive (that is, if $A,B \in \calF$ such that $A\cap B =\emptyset$ then $\mu (A \cup B) = \mu (A) + \mu (B)$) and 
	    \item $\mu$ is continuous from below (that is, if $\left\{ A_n \right\}$ is nondecreasing sequence of elements from $\calF$ then $\mu\left( \bigcup \left( A_i \right) \right) = \lim_{n\to\infty} \mu \left( A_n \right)$).
	    \end{enumerate}
    \label{thm:equiv-measure}
\end{theorem}
