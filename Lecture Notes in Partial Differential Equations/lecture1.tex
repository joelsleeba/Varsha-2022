\section{Lecture 1 --- 11th August, 2022 --- Definition, Classifications \& Examples of PDEs}



\subsection{Notations}
\begin{itemize}[label=--]
    \item Let $\N _0$ be defined to be the set $\N \cup \left\{ 0 \right\}$. For any $N\in \N$, an element of $\N _ 0 ^N$ to be an \textit{multiindex}. If $\alpha \in \N _0 ^N$ then $\alpha = \left( \alpha _1 ,\alpha_2,\ldots, \alpha_N \right)$ for some $\alpha_i \in \N _0$.

    \item For any $x\in \R ^N$ and $N\in \N$, we define $x^\alpha = x_1 ^{\alpha_1}\cdots x_N ^{\alpha _N}$.

    \item We will denote $\Omega \subset \R ^N$ to be an open subset.

    \item Given any multiindex $\alpha \in \N _0 ^N$, we define $|\alpha | =\alpha_1 + \alpha _2 + \ldots + \alpha _N$.

    \item Given any multiindex $\alpha$, we define
    \begin{equation*}
        D^{\alpha} := \frac{\partial ^{|\alpha|}}{\partial_{x_{1}} ^{\alpha_1} \cdots \partial_{x_{N}} ^{\alpha_N}} = \partial_{ x_1 }^{\alpha _1}\cdots \partial_{ x_N }^{\alpha_N}
    \end{equation*}

    $\ie$ for all $x\in \Omega$
    \begin{equation*}
        D^{\alpha}(u) := \frac{\partial ^{|\alpha|}(u)}{\partial_{x_{1}} ^{\alpha_1} \cdots \partial_{x_{N}} ^{\alpha_N}} = \partial_{ x_1 }^{\alpha _1}\cdots \partial_{ x_N }^{\alpha_N}(u)
    \end{equation*}

    \begin{remark}
    Note that it is by the \href{'https://calculus.subwiki.org/wiki/Clairaut's_theorem_on_equality_of_mixed_partials'}{Clairaut's theorem on equality of mixed partials} that we can club together the derivatives $\wrt$ one index without worrying about the order in which they are differentiated.
    \end{remark}
    \item For any $k\in \N$, denote $D^k =\left\{ D^{\alpha} : |\alpha| =k \right\}$
\end{itemize}



\subsection{Definition,  Classification \& Examples}
\begin{definition}[Partial Differential Equation]
    Let $\Omega$ be an open subset of $\R ^N$. An expression of the form  
        \begin{equation}
            F\left( D^{k} u(x), D^{k-1} u(x), \ldots , Du(x), x \right)=0 \qquad (x\in \Omega)
            \label{eqn:pde}
        \end{equation}

    is called a $k$th order PDE for the unknown function $u:\Omega \to \R$. One may assume $F: \R ^{N^k} \times \R^{N^{k-1}} \times \cdot \times \R ^N \times \Omega \to \R$ is a given smooth function.

    \begin{remark}
        Note that $u$ being a real valued function will be mapped into $\R$, while the $D(u)$ will have $k$ components each corresponding to the derivatives $\wrt$ to each index of the preimage $x$ of $u(x)$
    \end{remark}

\end{definition}



\subsubsection{Classifications of PDE}
\begin{enumerate}[label=(\roman*)]

    \item The PDE (\ref{eqn:pde}) is called \textit{linear} if it has the form $$\sum_{0\le |\alpha| \le k} a_{\alpha} (x) D^{\alpha} u =f$$ 
    \begin{flushright}
     $\ie$ each summand should have a degree less than $k$
    \end{flushright}
	for some functions $a_{ \alpha }$, f. The linear PDE is homogeneous if $f=0$.

    \item The PDE (\ref{eqn:pde}) is called \textit{semilinear} if it has the form
        $$\sum_{|\alpha|=k} a_{\alpha}(x) D^{\alpha}u + a_0 \left( D^{k-1}u, \ldots , Du, u, x \right) =0$$

    \item The PDE (\ref{eqn:pde}) is called \textit{quasilinear} if it has the form 
        $$\sum_{|\alpha =k} a_{\alpha} (D^{k-1} u, \ldots , Du, u, x) D^{\alpha} u + a_0 \left( D^{k-1} u, \ldots , Du, u, x \right) = 0$$

    \item The PDE (\ref{eqn:pde}) is called \textit{nonlinear} if the PDE has a nonlinear dependence on the highest order derivative.

\end{enumerate}

\begin{definition}[System of PDE]
An expression of the form $\mathbf{F} \left( D^k (\mathbf{u}), D^{k-1} (\mathbf {u})), \ldots , D (\mathbf{u}), \mathbf{u} , x \right)=\mathbf{0}$ is called a $k$th order system of PDE, where $\mathbf{u} : \Omega \to \R ^m$ is the unknown, $\mathbf{u}= \left( u^1 , u^2 , \ldots, u^n \right)$ and $\mathbf{F} : \R ^{mN^k} \times \R ^{mN^{k-1}} \times \cdots \R ^{mN} \times \R ^m \times U \to \R ^m$ is given.
    \label{def:system-of-pde}
\end{definition}



\subsubsection{Examples of PDEs}
\begin{enumerate}

    \item Linear Equations
	\begin{description}
	    \item[Laplace Equation] $\Delta u = \sum_{i=1}^{N} \partial _{{x_i}^{2}} u =0$ \begin{flushright}(linear, second order)\end{flushright}

	    \item [Linear Transport Equation] $\partial_t u + \sum_{i=1}^{N} \partial_{x_i} u =0$ \begin{flushright}(linear, first order)\end{flushright}

	    \item [Schrödinger's Equation] $i \partial_{t} u + \Delta u =0$ \begin{flushright}(linear, second order)\end{flushright}

	    \item [Linear System : Maxwell's Equations]
		\begin{align*}
		    \partial_t E &=\text{curl } B \\
		    \partial_t B &= - \text{curl } E \\
		    \text{div } E &= \text{div } B =0
		\end{align*}

	\end{description}

    \item Nonlinear equations
	\begin{description}
	    \item[Inviscid Burgers' equation] $\partial_{t} u + u \partial _x u = 0$
	    \item [Eikonal equation] $|Du| = 1$
	    \item [Nonlinear system: Navier-Stokes Equations] \begin{align*}\partial _t \mathbf{u} + \mathbf{u} \cdot D\mathbf{u} - \Delta \mathbf{u} &= -Dp  \\ \text{div } \mathbf{u} &=0\end{align*}
	\end{description}

\end{enumerate}

\begin{definition}[Well posed]
    A PDE is said to be \textit{well posed} if
    \begin{description}
	\item[(Existence)] it has at least one solution,
	\item[(Uniqueness)] it has at most one solution and
	\item[(Stability)] the solution depends continuously on the data given in the problem.
    \end{description}
    \label{def:well-posed}
\end{definition}

\begin{definition}
    A \textit{classical solution} of the $k$-th order PDE is a function $u\in C^k (\Omega)$ which satisfies the equation pointwise
    $$F\left( D^k u(x) , D^{k-1} u(x), \ldots , Du(x), u(x),x \right) =0$$
    for all $x\in \Omega$.
    \label{def:classical-solution}
\end{definition}

\begin{remark}
    A classical solution may not always exist. For instance, the inviscid Burgers' equation does not have a solution.
\end{remark}

The course is divided into three parts:
\begin{enumerate}[label=(\alph*)]
    \item Representation Formulae for solutions
    \item Linear PDE theory
    \item Nonlinear PDE theory
\end{enumerate}



\subsection{Transport Equation}

The PDE
\begin{equation*}
    \partial_t u + b \cdot Du = 0 \text{ in } \R ^n \times (0, \infty)
\end{equation*}
where $t\in (0, \infty ), x\in \R ^n$ are the independent variables, $u=u(t,x)$ is the dependent variable and $b=\left( b_1 ,b_2 , \ldots , b_n \right)$ and $Du = \left( \partial _ {x_1} u , \ldots , \partial _{x_2} u \right)$ is the gradient.

