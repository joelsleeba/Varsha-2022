\documentclass[12pt]{article}
\usepackage[margin=1in]{geometry}
\usepackage{amsfonts, amsmath}
\usepackage[T1]{fontenc}
\usepackage{mathrsfs, enumitem}
\usepackage{dirtytalk,hyperref}
\usepackage[utf8]{inputenc}
\usepackage{amssymb}
\usepackage{amsfonts}
\usepackage{amsmath}
\usepackage{amsthm}
\usepackage{color}
\usepackage{hyperref}
\usepackage{csquotes}
\usepackage{fourier}

\newtheorem{theorem}{Theorem}[subsection]
\newtheorem{lemma}[theorem]{Lemma}
\newtheorem{claim}[theorem]{Claim}
\newtheorem{proposition}[theorem]{Proposition}
\newtheorem{corollary}[theorem]{Corollary}
\newtheorem{fact}[theorem]{Fact}
\newtheorem{notation}[theorem]{Notation}
\newtheorem{observation}[theorem]{Observation}
\newtheorem{conjecture}[theorem]{Conjecture}

\theoremstyle{definition}
\newtheorem{definition}[theorem]{Definition}
\newtheorem{example}[theorem]{Example}

\theoremstyle{remark}
\newtheorem{remark}[theorem]{Remark}
\theoremstyle{plain}
\newcommand{\ignore}[1]{}

% \renewcommand{\Pr}{{\bf Pr}}
% \newcommand{\Prx}{\mathop{\bf Pr\/}}
% \newcommand{\E}{{\bf E}}
% \newcommand{\Ex}{\mathop{\bf E\/}}
% \newcommand{\Var}{{\bf Var}}
% \newcommand{\Varx}{\mathop{\bf Var\/}}
% \newcommand{\Cov}{{\bf Cov}}
% \newcommand{\Covx}{\mathop{\bf Cov\/}}

% shortcuts for symbol names that are too long to type
\newcommand{\eps}{\epsilon}
\newcommand{\lam}{\lambda}
\renewcommand{\l}{\ell}
\newcommand{\la}{\langle}
\newcommand{\ra}{\rangle}
\newcommand{\wh}{\widehat}
\newcommand{\wt}{\widetilde}

% % "blackboard-fonted" letters for the reals, naturals etc.
\newcommand{\R}{\mathbb R}
\newcommand{\N}{\mathbb N}
\newcommand{\Z}{\mathbb Z}
\newcommand{\F}{\mathbb F}
\newcommand{\Q}{\mathbb Q}
\newcommand{\C}{\mathbb C}

% % operators that should be typeset in Roman font
% \newcommand{\poly}{\mathrm{poly}}
% \newcommand{\polylog}{\mathrm{polylog}}
% \newcommand{\sgn}{\mathrm{sgn}}
% \newcommand{\avg}{\mathop{\mathrm{avg}}}
% \newcommand{\val}{{\mathrm{val}}}

% % complexity classes
% \renewcommand{\P}{\mathrm{P}}
% \newcommand{\NP}{\mathrm{NP}}
% \newcommand{\BPP}{\mathrm{BPP}}
% \newcommand{\DTIME}{\mathrm{DTIME}}
% \newcommand{\ZPTIME}{\mathrm{ZPTIME}}
% \newcommand{\BPTIME}{\mathrm{BPTIME}}
% \newcommand{\NTIME}{\mathrm{NTIME}}

% values associated to optimization algorithm instances
\newcommand{\Opt}{{\mathsf{Opt}}}
\newcommand{\Alg}{{\mathsf{Alg}}}
\newcommand{\Lp}{{\mathsf{Lp}}}
\newcommand{\Sdp}{{\mathsf{Sdp}}}
\newcommand{\Exp}{{\mathsf{Exp}}}

% if you think the sum and product signs are too big in your math mode; x convention
% as in the probability operators
\newcommand{\littlesum}{{\textstyle \sum}}
\newcommand{\littlesumx}{\mathop{{\textstyle \sum}}}
\newcommand{\littleprod}{{\textstyle \prod}}
\newcommand{\littleprodx}{\mathop{{\textstyle \prod}}}

% horizontal line across the page
\newcommand{\horz}{
\vspace{-.4in}
\begin{center}
\begin{tabular}{p{\textwidth}}\\
\hline
\end{tabular}
\end{center}
}

% calligraphic letters
\newcommand{\calA}{{\cal A}}
\newcommand{\calB}{{\cal B}}
\newcommand{\calC}{{\cal C}}
\newcommand{\calD}{{\cal D}}
\newcommand{\calE}{{\cal E}}
\newcommand{\calF}{{\cal F}}
\newcommand{\calG}{{\cal G}}
\newcommand{\calH}{{\cal H}}
\newcommand{\calI}{{\cal I}}
\newcommand{\calJ}{{\cal J}}
\newcommand{\calK}{{\cal K}}
\newcommand{\calL}{{\cal L}}
\newcommand{\calM}{{\cal M}}
\newcommand{\calN}{{\cal N}}
\newcommand{\calO}{{\cal O}}
\newcommand{\calP}{{\cal P}}
\newcommand{\calQ}{{\cal Q}}
\newcommand{\calR}{{\cal R}}
\newcommand{\calS}{{\cal S}}
\newcommand{\calT}{{\cal T}}
\newcommand{\calU}{{\cal U}}
\newcommand{\calV}{{\cal V}}
\newcommand{\calW}{{\cal W}}
\newcommand{\calX}{{\cal X}}
\newcommand{\calY}{{\cal Y}}
\newcommand{\calZ}{{\cal Z}}

% bold letters (useful for random variables)
%----------------------------------------------
% \renewcommand{\a}{{\boldsymbol a}}
% \renewcommand{\b}{{\boldsymbol b}}
% \renewcommand{\c}{{\boldsymbol c}}
% \renewcommand{\d}{{\boldsymbol d}}
% \newcommand{\e}{{\boldsymbol e}}
% \newcommand{\f}{{\boldsymbol f}}
% \newcommand{\g}{{\boldsymbol g}}
% \newcommand{\h}{{\boldsymbol h}}
% \renewcommand{\i}{{\boldsymbol i}}
% \renewcommand{\j}{{\boldsymbol j}}
% \renewcommand{\k}{{\boldsymbol k}}
% \newcommand{\m}{{\boldsymbol m}}
% \newcommand{\n}{{\boldsymbol n}}
% \renewcommand{\o}{{\boldsymbol o}}
% \newcommand{\p}{{\boldsymbol p}}
% \newcommand{\q}{{\boldsymbol q}}
% \renewcommand{\r}{{\boldsymbol r}}
% \newcommand{\s}{{\boldsymbol s}}
% \renewcommand{\t}{{\boldsymbol t}}
% \renewcommand{\u}{{\boldsymbol u}}
% \renewcommand{\v}{{\boldsymbol v}}
% \newcommand{\w}{{\boldsymbol w}}
% \newcommand{\x}{{\boldsymbol x}}
% \newcommand{\y}{{\boldsymbol y}}
% \newcommand{\z}{{\boldsymbol z}}
% \newcommand{\A}{{\boldsymbol A}}
% \newcommand{\B}{{\boldsymbol B}}
% \newcommand{\C}{{\boldsymbol C}}
% \newcommand{\D}{{\boldsymbol D}}
% \newcommand{\E}{{\boldsymbol E}}
% \newcommand{\F}{{\boldsymbol F}}
% \newcommand{\G}{{\boldsymbol G}}
% \renewcommand{\H}{{\boldsymbol H}}
% \newcommand{\I}{{\boldsymbol I}}
% \newcommand{\J}{{\boldsymbol J}}
% \newcommand{\K}{{\boldsymbol K}}
% \renewcommand{\L}{{\boldsymbol L}}
% \newcommand{\M}{{\boldsymbol M}}
% \renewcommand{\O}{{\boldsymbol O}}
% \renewcommand{\P}{{\mathbb{P}}}
% \newcommand{\Q}{{\boldsymbol Q}}
% \newcommand{\R}{{\boldsymbol R}}
% \renewcommand{\S}{{\boldsymbol S}}
% \newcommand{\T}{{\boldsymbol T}}
% \newcommand{\U}{{\boldsymbol U}}
% \newcommand{\V}{{\boldsymbol V}}
% \newcommand{\W}{{\boldsymbol W}}
% \newcommand{\X}{{\boldsymbol X}}
% \newcommand{\Y}{{\boldsymbol Y}}
% \newcommand{\Z}{{\boldsymbol Z}}



\title{Lecture Notes in Partial Differential Equations}
\author{Ashish Kujur}
\date{June 2022}
\begin{document}

\maketitle
\section*{Introduction}
This is a set of lecture notes which I took for reviewing stuff that I typed after taking class from Professor \textit{insert name here}. All the typos and errors are of mine. I like to take notes in \LaTeX as it motivates me to drag my ass to class. The pictures that make here will be hand drawn and I will appreciate it if someone who is knowledgeable in Tikz will help me digitizing my rough hand-drawn pictures.
\tableofcontents

%\section{Lecture 1 --- 10th August 2022 --- Local Rings, Semilocal rings, Chinese Remainder Theorem}

We will be assuming the following things before proceeding in the course:
\begin{itemize}
    \item A ring $A$ is a commutative ring with unity.
    \item Existence of maximal ideals in a commutative ring with unity (this follows immediately from Zorn's Lemma)
    \item Definition of ring morphism.
    \item Definition of prime and maximal ideals and the facts that
	\begin{itemize}
	    \item $P$ is a prime ideal of $A$ iff $A/P$ is an integral domain and
	    \item $M$ is a maximal ideal of $A$ iff $A/P$ is a field
	\end{itemize}
\end{itemize}

\subsection{Basic Definitions --- Local Rings, Semilocal rings and few other results}

\begin{definition}[local ring]
    Let $A$ be a ring. $A$ is said to be a \textit{local ring} if $A$ has a unique maximal ideal $M$. A local ring is often denoted by $(A,M)$. 
    \label{def:local-ring}
\end{definition}

\begin{definition}[semilocal ring]
    Let $A$ be a ring. $A$ is said to be \textit{semilocal ring} if $A$ has only fintiely many maximal ideals.
    \label{def:semilocal-ring}
\end{definition}

How does one come up with a semilocal ring with exactly $m$ maximal ideals? Here's an example:
\begin{example}[A ring with $m$ distinct maximal ideals]
    Let $A= \Z / n\Z$. It is fairly easy to show that all the ideals of $A$ are of the form $\left( \overline k \right)$ where $k\in\N$ and $k\mid n$ and also that if $k,j \mid n$ and $\left( \overline k \right) \subset \left( \overline j \right)$ iff $j\mid k$. (See Sepanski Exercise 3.47 and 3.48) Now let $p_1 , p_2 , \ldots ,  p_m$ be $m$ distinct primes. Define $n=p_1 p_2 \cdots p_m$. It is easy to see from the aforementioned facts that $A = \Z / n\Z$ has $m $ distinct maximal ideals.
\end{example}


\begin{example}[A standard example of a local ring?]

    Let $A$ be a ring, $M$ be a maximal ideal of $A$ and $n\in \N$. Observe that $M^n$ is a ideal of $A$ (See Sepanski Exercise 3.51). We claim that $A/ M^n$ has only prime ideal namely $M / M^n$. Let $\calP$ be a prime ideal of $A / M^n$. Then by the correspondence theorem, $\calP = P/ M^n$ where $P$ is a prime ideal of $A$ containing $M^n$. Then $P\supset M^n$ which further implies that $P \supset M$ (due to Lemma \ref{lemma:equiv-prime}. Since $M$ is a maximal ideal, we have that $P=M$. This completes the proof of the claim. Also, note that since every maximal ideal is prime, we have that $A/M^n$ is a local ring.

\end{example}
 
\begin{fact}
    Let $A$ be ring, $B$ be an integral domain, $f: A\to B$ be a ring morphism and $Q$ be a prime ideal of $B$. Then $\ker (f)$ is a prime ideal of $A$.
    \label{fact:inverse-prime}
\end{fact}
\begin{proof}[Proof of the fact]
    Suppose that $ab \in \ker (f)$. Then $f(ab)=0$ which further implies $f(a)f(b)=0$ and hence $a\in \ker(f)$ or $b\in\ker(f)$ since $B$ is an integral domain.
\end{proof}

\begin{lemma}
    Let $A, B$ be rings, $f:A\to B$ be a ring morphism and $Q$ be a prime ideal in $B$. Then $f^{-1} \left( Q \right)$ is a prime ideal of $A$. 
    \label{lemma:inverse-image-of-prime-ideal}
\end{lemma}
\begin{proof}
    Let $p: B \to B/Q$ be the canonical homomorphism. Consider the map $p \circ f : A \to B/Q$. We show that $\ker (p \circ f) = f^{-1} \left( Q \right)$. The lemma will follows from fact \ref{fact:inverse-prime}, if we show that $\ker (p \circ f) = f^{-1}\left( Q \right)$ as $B/Q$ is an integral domain. So consider the following:
    \begin{align*}
	x \in \ker \left( p \circ f \right) & \Leftrightarrow p (f(x)) = Q \\
					& \Leftrightarrow f(x) + Q = Q \\
					& \Leftrightarrow f(x) \in Q \\
					& \Leftrightarrow x \in f^{-1} \left( Q \right)
    \end{align*}
\end{proof}

% complete the commutative diagram here using Tikz and show that above proof holds in general.

\begin{lemma}
    Let $A$ be a ring, let $I,J $ be ideals of $A$ and $P$ be a prime ideal of $A$. If $P \supset IJ$ then either $P \supset I$ or $P\supset J$.
    \label{lemma:equiv-prime}
\end{lemma}
\begin{proof}
    Suppose that $P \not \supset I$. Then there is some $i\in I \setminus P$. We show that $J \subset P$. Let $j \in P$. Then $ij \in IJ$ and hence $ij \in P$. Since $P$ is a prime ideal, we must have that either $i\in P $ or $j \in P$. But the former is not possible by assumption, therefore, $j\in P$. Since $j$ was arbitrary, the proof is complete.
\end{proof}

\begin{remark}
    Let $A$ be a ring, $I$ be any ideal of $A$. Then there is a maximal ideal $M$ of $A$ containing $A$. The proof of this remark is fairly straightforward. Consider the ring $A/I$. Since every ring has a maximal ideal, so there must be some maximal ideal $\calM$ of $A/I$. By the correspondence theorem, $\calM = M/I$ for some ideal $M$ of $A$. This ideal $M$ of $A$ must be maximal again by the correspondence theorem and this completes the proof of the remark.
    \label{rem:maximal-ideal-containing-some-ideal}
\end{remark}

\begin{lemma}
    Let $A$ be a ring, $I, J, K$ be ideals of $A$. Furthermore, assume that $I, J$ are comaximal and $I, K$ are comaximal. Then $I+JK=A$. (Recall that two ideals $I,J$ are said to be comaximal if $I+J=A$.)
    \label{lemma:comaximal}
\end{lemma}
\begin{proof}
    Suppose that $I+JK \subsetneq A$. Then by Remark \ref{rem:maximal-ideal-containing-some-ideal}, we have that there is some maximal (and hence prime) ideal $P$ containing $I+JK$. Thus, we have that $I\subset P$ and $JK \subset P$. 

    From $JK \subset P$, we can conclude that $J\subset P$ or $K \subset P$ from Lemma \ref{lemma:equiv-prime}. But in the either case, we have that $I+J \subset P \subsetneq A$. A contradiction and hence $I+JK = A$.
\end{proof}

\begin{example}
    Let $A=\Z$. Note that the ideal $\left( 3,4 \right)$ generated by $3$ and $4$ and the ideal $\left( 3,5 \right)$ generated by $3$ and $5$ are exactly $\Z$. Thus, the ideal $\left( 3, 20 \right) =A$ by Lemma \ref{lemma:comaximal}.
\end{example}

\subsection{Chinese Remainder Theorem}

\begin{theorem}[Chinese Remainder Theorem]
    Let $A$ be a ring, $I_1 , I_2 , \ldots , I_n$ be ideals of $A$. Consider the canonical map $\varphi : A \to A/I_1 \times A/I_2 \times \cdots A/I_n$ given by $\varphi \left( x \right) = \left( x+I_1 , \ldots , x+I_n \right)$. Then the following holds:
    \begin{enumerate}
	\item If $I_p , I_q$ are comaximal for all $1 \le p < q \le n$ then $I_1 I_2 \ldots I_n = I_1 \cap I_2 \cap \ldots \cap I_n$
	\item $\varphi$ is injective iff $\ker \varphi = I_1 \cap I_2 \cap \ldots \cap I_n = \left\{ 0 \right\}$
	\item If $\varphi$ is surjective iff $I_m , I_n$ are comaximal for all $1\le m < n \le n$
    \end{enumerate}
    \label{thm:CRT}
\end{theorem}

\begin{proof}[Proof of (1)]
    We proceed by induction on $n$. Suppose that $n=2$. Consider the ideals $I_1 , I_2$ satisfying $I_1 + I_2 = A$. We show that $I_1 I_2 = I_1 \cap I_2$.

    It is fairly easy to see that $I_1 I_2 \subset I_1 \cap I_2$. if $i_1 \in I_1$ and $i_2 \in I_2$ then $i_1 i_2 \in I_1$ and $i_1 i_2 \in I_2$ as $I_1$ and $I_2$ are both ideals of $A$. Hence, $i_1 i_2 \in I_1 \cap I_2$. To see the reverse inclusion, we use the comaximality of $I_1$ and $I_2$. Since $I_1 + I_2 =A$, $1=i_1 + i_2$ for some $i_1 \in I_1$ and some $i_2 \in I_2$. Let $c \in I_1 \cap I_2$. Then $c= i_1 c + c i_2 $. Clearly $i_1 c \in I_1 I_2$ and $c i_2 \in I_1 I_2$ and hence $c\in I_1 I_2$.

    Suppose that (1) holds true for any $n-1$ ideals of $A$ where $n>2$. Let $I_1 , I_2 , \ldots , I_n$ be ideals of $A$. Define $J=I_1 I_2 \cdots I_{n-1}$ and $I= I_n$. We show that $I+J=A$.

    It is easy to see that $I+J \subset A$. Now we use that comaximality of $I_{n-1}$ and $I_n$. By the comaximality, we have $1 = i_{n-1} + i_{n}$ for some $i_{n-1}\in I_{n-1}$ and some $i_n \in I_{n}$. Let $a\in A$. Then $a=ai_{n-1} + ai_{n}$. Clearly, $ai_{n} \in I_{n}$ as $I_{n}$ is an ideal and $ai_{n-1} \in I_{n-1}$. Since $I_{n-1}\subset I$, we are done.

    By the $n=2$, it follows that $IJ=I\cap J$. Now our result follows from the induction hypothesis:
    \begin{align*}
	I_1 \ldots I_{n-1} I_n &= JI \\
	                       &= J \cap I \\
			       &= I_1 \ldots I_{n-1} \cap I_n \\
			       &= I_1 \cap \ldots \cap I_{n-1} \cap I_n
    \end{align*}
    Observe that the third equality follows from the induction hypothesis.
    
\end{proof}


%\section{Lecture 2}
\subsection{Lebesgue Integral}

\begin{definition}[Function]
    \label{<+label+>}
    Let $X,Y$ be two sets. A function $f:X\to Y$ is a object which takes every element $x\in X$ to a unique element $y\in Y$ such that $f(x)=y$.
\end{definition}



\end{document}
