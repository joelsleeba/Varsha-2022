\section{Lecture 2 --- 16th August, 2022 --- Linear Transport Equation}

Consider the linear transport equation given by 
$$u_t + b\cdot Du = 0 \qquad \text{ in } \R ^n \times (0, \infty)$$
where $b\in \R ^n$ is a fixed vector and $u: \R ^n \times [0, \infty ) \to \R$ is the unknown function, $Du = D_x u (u_{x_1}, u_{x_2} , \ldots , u_{x_n})$. 

Let $(x,t) \in \R ^n \times (0, \infty)$ be fixed. Our aim is to obtain a representation of the solution.

Assuming that $u$ is a smooth function, we define the function $z(s):= u(x+bs,t+s)$ where $s\in \R$.

Note that $z$ is the restriction of the function $u$ to the line $L=\left\{ \left( x+bs, t+s  \right) : s \in \R \right\}$. Note that the line $L$ passes through $(x,t)$ and is in the direction of the vector $(b,1)$.

Differentiating $z$, we have that for all $s\in \R$, 
\begin{equation*}
    z'(s)=b\cdot Du (x+bs , t+s) + u_t (x+bs, t+s) = 0 
\end{equation*}

Thus, $u$ is a constant on the line $L$. If we know the solution at any point on $L$, the problem is solved. We use the aforementioned result in the following subsection.

\subsection{Solution of an IVP}

Let $g: \R ^n \to \R$ where $g=g(x_1, x_2 ,\ldots , x_n)$. We consider the following IVP
\begin{align*}
    u_t + b\cdot Du &= 0 \quad \text{ in } \R ^n \times (0,\infty ) \\
    u &= g \quad \text{ in } \R^n \times \left\{ 0 \right\}
\end{align*}

Just note that the last equation in IVP means $u(x,0) = g(x)$ for all $x\in \R ^n$. From the discusssion before the start of the subsection, it suffices to know the solution on the hyperplane $\Gamma = \R ^n \times \left\{ 0 \right\}$. The line $L$ passes through $\Gamma$ at the point $(x-tb, 0 )$. So,

\begin{align*}
    u(x,t)&= z(0) \\
    &= z(-t) \\
&= u(x-bt, 0) \\
&= g(x-bt)
\end{align*}

Note that the first equality is true by definition of $z$, the second equality is true by $z$ being constant on the line $L$ and the third is again true by definition of $z$ and the last is true by $u=g$ in $\R ^n \times \left\{ 0 \right\}$.

So, \boxed{u(x,t) = g(x-bt) \qquad t \ge 0, x\in \R ^n}.

\begin{remark}
    Draw the $x$ versus $u$ sketch when $n=1$ and taking $t=0$ and $t=1$ and notice the solution gets translated or transported and hence the name of the PDE is linear transport equation.
\end{remark}

\subsection{Solution of a homogeneous problem}
Consider the problem
\begin{align*}
    u_t + b\cdot Du &= f \quad \text{ in } \R ^n \times (0,\infty ) \\
    u &= g \quad \text{ in } \R^n \times \left\{ 0 \right\}
\end{align*}
 
Take any $\left( x,t \right) \in \R ^n \times \left( 0, \infty \right)$ and define $z(s)= u (x+sb, t+s)$.

Hence, $z'(s) = u_t (x+sb, t+s ) + b \cdot Du (x+sb, t+s) = f(x+sb , t+s )$. So,

\begin{align*}
    u(x,t)-g(x-tb) &= z(0) - z(-t) \\
    &= \int_{-t}^{0} z'(s) ds \\
    &= \int_{-t}^{0} f\left( x+sb, t+s \right) ds \\
    &= \int_{0}^{t} f\left( x+(s-t)b, s \right) ds
\end{align*}
Note that the last equality is by change of variable. Therefore, we have 
\begin{equation*}
    \boxed{u(x,t)=g(x-tb)+\int_{0}^{t} f\left( x+(s-t)b, s\right) ds} \quad (x\in \R ^n , t \ge 0)
\end{equation*}
is the required solution.


