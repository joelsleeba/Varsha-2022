\section{Lecture 2 --- 12th August, 2022 --- A hell lot of definitions! (and some examples)}

\subsection{Some definition and remarks on the definition of Frames}

\begin{enumerate}
    \item In \ref{eqn:def-frame}, $A,B$ are called the frame bounds.
    \item The infimum (corr. supremum) over all the upper (corr. lower) frame bounds is called the optimal upper (corr. lower) frame bound.
    \item The optimal frame bounds are also frame bounds! We can verify this in this fashion: Let $\beta$ be the optimal upper frame bound. Note that in equation (\ref{eqn:def-frame}) holds trivially for $f=0$. So, we need to only consider the case when $f\ne 0$. Let $B$ be any upper frame bound and $f$ be any nonzero vector. Then $\sum_{i} \norm{\ip{f}{f_i}}^2 \le B \norm{f}^2$. Since $f\ne 0$ by our choice,  $\sum_{i} \norm{\ip{f}{f_i}}^2 / \norm{f}^2 \le B$. Since $B$ is arbitrary, we have that $\sum_{i} \norm{\ip{f}{f_i}}^2 / \norm{f}^2 \le \beta$. Now since $f$ was arbitrary, out result follows. The almost same proof works for optimal lower bound as well!
    \item If $A=B$ in equation \ref{eqn:def-frame} then frame is called a \textit{tight frame}.
    \item If $A=B=1$ in equation \ref{eqn:def-frame} then the frame is called a \textit{Parseval frame} (as it will then satisfy the Parseval's identity).
    \item A frame $\left\{ f_i \right\}$ is called \textit{equiangular} if there is a constant $C$ such that $|\ip{f}{f_i}|=C$ for all $i\ne j$.
    \item A frame $\left\{ f_i \right\}_{i\in I}$ is called \textit{equal norm frame} if there is a constant $C$ such that $\norm{f_i}=C$ for all $i\in I$.
    \item A frame $\left\{ f_i \right\} _{i\in I}$ is called a \textit{exact frame} if for any $j\in I$, $\left\{ f_i \right\} _{i \in I\setminus \{ j \}}$ is no longer a frame!
    \item Let $\left\{ f_i \right\}_{i\in I}$ be a frame and $x\in H$. Then the values $\left\{ \ip{x}{f_i} \right\}_{i\in I}$ are called the \textit{frame coefficients} of $x$.
    \item A sequence $\left\{ f_i \right\}_{i=1}^{N}$ is called a \textit{Bessel sequence} if there is a positive constant $ B$ such that $\sum_{i=1}^{N} |\ip{f}{f_i}| \le B \norm{f} ^2$ for all $f\in H$. 

\end{enumerate}

\subsection{Examples of Frames}
In the following examples, let $\left\{ e_1 , e_2 , \ldots , e_n \right\}$ be orthonormal basis for $H$.
\begin{enumerate}
    \item Consider the list \\
	$\left\{ e_1 / \sqrt{2} , e_1 / \sqrt{2} , e_2 / \sqrt{2} , e_2 / \sqrt{2}, \ldots , e_n / \sqrt{2} , e_n / \sqrt{2} \right\}$. Then
	\begin{align*}
	    \sum_{i=1}^{N} |\ip{f}{f_i}|^2 &= \frac{1}{2} \sum_{i=1}^{n} |\ip{f}{e_i}|
	+ \frac{1}{2} \sum_{i=1}^{n} |\ip{f}{e_i}| \\
	&= \sum_{i=1}^{n} |\ip{f}{e_i}|^2 = \norm{f}
    \end{align*}
    The last equality holds because $\left\{ e_i \right\}_{i=1}^{n} $ is an orthonormal basis. So the aforementioned list of vectors is a Parseval frame.
    
\item Consider the list $\left\{ e_1, e_1, e_2 , \ldots, e_n \right\}$. Then
    \begin{align*}
	\sum_{i=1}^{N} |\ip{f}{f_i}|^2 &= |\ip{f}{e_1}|^2 + \sum_{i=1}^{n} |\ip{f}{e_i}|^2 \\
	&\le \norm{f}^2 + \norm{f}^2 \\
	&= 2\norm{f}^2
    \end{align*}
    and 
\begin{align*}
	\sum_{i=1}^{N} |\ip{f}{f_i}|^2 &= |\ip{f}{e_1}|^2 + \sum_{i=1}^{n} |\ip{f}{e_i}|^2 \\
	&\ge \norm{f}^2
    \end{align*}
    Thus, $\left\{ e_1, e_1 , e_2 , \ldots , e_n \right\}$ is a frame for $H$ with frame bounds $1$ and $2$. In fact the frame bounds are optimal, consider $f=e_1$ and $f=e_2$ and note the frame bounds are actually achieved!

\item Consider $\left\{ e_1, e_1 , e_2, e_2 , \ldots , e_n , e_n \right\}$. Then this is a tight frame bound with bound $2$.
\end{enumerate}

\subsection{Properties of frames in finite dimensional Hilbert spaces}

Note that the following lemma only holds for finite dimensional Hilbert space $H$.

\begin{lemma}
    Let $\left\{ f_i \right\}_{i\in I}$ be a family of vectors in $H$. Then
    \begin{enumerate}[label=(\arabic*)]
	\item If $\left\{ f_i \right\}_{i\in I}$ is an orthonormal basis, then $\left\{ f_i \right\}_{i\in I}$ is a Parseval frame but the converse may not be true!
	\item $\left\{ f_i \right\}_{i\in I}$ is a frame for $H$ iff $\text{span } \left\{ f_i \right\} =H$
	\item If $\left\{ f_i \right\}$ is a unit norm Parseval frame iff $\left\{ f_i \right\}_{i\in I}$ is an orthonormal basis for $H$.
	\item $\left\{ f_i \right\}$ is exact then $\left\{ f_i \right\}$ is linearly independent.
    \end{enumerate}
    \label{lemma:prop-frames}
\end{lemma}

