\section{Lecture 1 --- 10th August, 2022 --- Hilbert Spaces \& Frames}
We start by reviewing the elementary notions from Linear Algebra.
\subsection{Inner Product Spaces}
\begin{definition}
    A vector space $V$ over a field $F$ ( $\R$ or $\C$ ) is called an \textit{inner product space} if there exists a function $\langle \cdot , \cdot \rangle : V \times V \to F$ satisfying the following:
    \begin{enumerate}
	\item $\langle x , x \rangle \ge 0$ for all $x\in V$.
	\item $\langle x , x \rangle =0$ iff $x=0$.
	\item (linear in the first argument) for all $x,y,z \in V$, $\langle x+y , z\rangle = \langle x , z \rangle + \langle y , z \rangle$
	\item (conjugate) for all $x,y \in V$, $\langle x , y \rangle = \overline{\langle y , x \rangle}$
    \end{enumerate}
    \label{def:ips}
\end{definition}

\begin{definition}
    A norm on a vector space $V$ over $\R$ or $\C$ is a function $\norm{\cdot} : V \to [0, +\infty)$ satisfying
    \begin{enumerate}
	\item for all $x\in V$, $\norm{x} = 0 \Leftrightarrow x=0$
	\item $\norm{x+y} \le \norm{x} + \norm{y}$ for all $x,y\in V$
	\item $\norm{\alpha x} =|\alpha| \norm{x}$ for all $x \in V$ and $\alpha \in F$
	\end{enumerate}
    \label{def:norm}
\end{definition}

It is easy to check that if $V$ is an inner product space then $\norm{\cdot}$ defined by $\norm{x}=\sqrt{\ip{x}{x}}$ is a norm on $V$. To verify the triangle inequality, use Cauchy Schwarz inequality.

\begin{definition}
    A vector space together with a norm is called a normed linear space.
    \label{def:normed-space}
\end{definition}

Note that every normed linear space $\left( V, \norm{\cdot} \right)$ is a metric space. The metric is given by $d (x, y) = \norm{x-y}$ for all $x,y \in V$.

\subsection{Hilbert Spaces \& Frames}

\begin{definition}
    An inner product space which is complete wrt the induced norm is called Hilbert Space.
    \label{def:Hilbert-Space}
\end{definition}

We will only be considering finite dimensional Hilbert spaces in this course!

\begin{example}
    \begin{enumerate}
	\item $\R ^n$ with the usual inner product is a Hilbert Space.
	\item $\C ^n$ with the usual inner product is a Hilbert Space.
	\end{enumerate}
\end{example}

\begin{definition}
    A sequence $\left\{ f_n \right\}$ in $H$ is called a frame for $H$ if there exists positive constants $A$ and $B$ such that
    $$A\norm{f}^2 \le \sum_{i} |\ip{f}{f_i}|^2 \le B \norm{f}^2$$
    for all $f\in H$.
    \label{def:frame}
\end{definition}

\begin{remark}
    It is possible to have that a frame in a finite dimensional Hilbert space consisting of infinitely many elements. However, it is rather artificial to have infinite number of frame elements in a finite dimensional space. We therefore consider only frames with finite number of elements.
    \label{rem:finite-frames}
\end{remark}


