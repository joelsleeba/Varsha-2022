\section{Lecture 3 --- 17th August, 2022 --- Proof of Prime Avoidance, Jacobson Radical, Modules}
\subsection{Proof of Prime Avoidance Lemma}
\begin{proof}[Proof of \ref{lemma:prime-avoidance}]
    We prove that the following equivalent statement that:
    
    If for all $j$, $I\not\subset P_j$ for all $j$, there is some element $x\in I$ such that $x\not\in P_j$ for all $j$. $(\star)$


    We prove this theorem by assuming that all but $2$ of the $P_i$ are prime ideals. (Note : this is a slightly weaker assumption!)
    
     We now start the proof using induction.
    
    We first consider the case when $n=2$. Let $I$ be an ideal and $P_1$ and $P_2$ be prime ideals of $A$ such that $I \not \subset P_1$ and $I \not \subset P_2$. So, there are some element $x\in I \setminus P_1$ and $y\in I \setminus P_2$.

    If $x\not \in P_2$ then we are done. Likewise if $y\not \in P_1$ then we are again done. So, we may assume that $x\in P_2$ and $y\in P_1$.

    Now consider $x+y$. Undoubtedly, $x+y \in I$. If it were the case that $x+y \in P_1$ then $x\in P_1$ which is not possible by choice of $x$. Likewise if it were the case that $x+y \in P_2$ then $y \in P_2$ as $x\in P_2$ which again is not possible by choice of $y$. Therefore, we have that $x+y \in I$, $x+y \not \in P_1$ and $x+y \not \in P_2$ and this ends our verification of the base case.

    Now, suppose that the $(\star)$ is true when the number of prime ideals is equal to $n-1$ where $n\ge 3$. 

    Let $I$ be an ideal and $P_1 , P_2 , \ldots , P_n$ be prime ideals such that $I \not \subset P_j$ for $1\le j \le n$. 

    By using the induction hypothesis, there is an element $x\in I$ such that $x \not \in P_j$ for $1 \le j \le n-1$.


    If $x\not\in P_n$ then our proof is complete! So, we assume that $x\in P_n$.
    
    Furthermore, we may assume that for $i\ne j$, it is not the case that $P_i \subset P_j$ or $P_j \subset P_i$, that is, there are no inclusions among the prime ideals. Since $n\ge 3$ and all but $2$ of the $P_j$ are prime ideals, we may assume that $P_n$ is a prime ideal.
    
    We claim that $IP_1 P_2 \ldots P_{n-1} \not\subset P_n$. Suppose not then $I P_1 P_2 \ldots P_{n-1} \subset P_n$. It follows by induction and Lemma \ref{lemma:equiv-prime} that $I \subset P_n$ or $P_i \subset P_n$ for some $1\le j \le n-1$. Note that the latter part of the 'or' cannot hold by our assumption in the previous paragraph. Thus $I \subset P_n$. But then again this is a contradiction! So, we have that $IP_1 P_2 \ldots P_{n-1} \not\subset P_n$.

    Now select a $y\in IP_1 \ldots P_{n-1}$ but $y\not \in P_n$.

    Now, we finish the proof by showing that $x+y \in I$ but $x+y \not \in P_i$ for all $1\le i \le n$. It is evident that $x+y \in I$. If $x+y\in P_n$ then $y\in P_n$ which is not possible by choice of $y$. Note that $y\in IP_1\ldots P_{n-1}$ implies $y\in P_i$ for all $1\le i\le n-1$. Now if $x+y \in P_i$ for some $1\le i \le n-1$ then $x \in P_i$. But that cannot happen by choice of $x$. Thus we have found an element which is in $I$ but not in any of $P_i$ and this completes the proof!

\end{proof}

\subsection{Jacobson Radical \& Local Rings revisited}

\begin{notation}
    Let $A$ be a ring. We will use $\text{max-spec} (A)$ to denote the set of all maximal ideals of $A$.
\end{notation}


\begin{definition}[Jacobson Radical]
    Let $A$ be a ring. The Jacobson radical $\calJ (A)$ is defined to by the intersection of all maximal ideals of $A$. In other words,
    \begin{equation*}
	\calJ (A) := \bigcap \left\{ m : m \in \text{max-spec} (A) \right\}
    \end{equation*}
    \label{def:Jacobson}
\end{definition}

\begin{lemma}
    Let $A$ be a ring. Then $x\in \calJ (A)$ iff $1-xy$ is a unit for all $y\in A$.
    \label{lemma:equiv-Jacobson}
\end{lemma}
\begin{proof}
    $\left( \Longrightarrow \right)$ Suppose that $x \in \calJ(A)$. Suppose that $1-xy$ is not a unit for some $y\in A$. Then there is some maximal ideal $m$ of $A$ containing $1-xy$. (Just consider the ideal generated by $1-xy$ and Remark \ref{rem:maximal-ideal-containing-some-ideal})

    Since $x\in \calJ(A)$, $x\in m$. So $xy\in m$ as $m$ is an ideal. Then $1= (1-xy)+xy \in m$ but this is not possible as maximal ideals are not the entire ring by definition! Hence $1-xy $ is a unit for all $y\in A$.

    $\left( \Longleftarrow \right)$ Now suppose that $1-xy$ is a unit for all $y\in A$. If $x\not \in \calJ (A)$ then there must be some maximal ideal $m$ of $A$ such that $x\in A\setminus m$. Now consider the ideal $m + (x)$. Clearly $m + (x) \supsetneq m$ for otherwise $x\in m$. Hence $m+ (x)=A$ as $m$ is a maximal ideal. Thus there are some elements $z\in m$ and $y \in A$ such that $z+xy =1$. But then $1-xy = z \in m$. Also, $1-xy$ is a unit, but that cannot possibly happen as maximal ideals cannot contain units!
\end{proof}

\begin{lemma}
    Let $A$ be a ring and $m$ be a nontrivial ideal such that every element of $A\setminus m$ is a unit. Then $\left( A, m \right)$ is a local ring.
    \label{lemma:sufficient-local}
\end{lemma}
\begin{proof}
    Let $I$ be any nontrivial ideal of $A$. To show that $\left( A,m \right)$ is a local ring, it suffices to show that that $I \subset m$. Let $x\in I$. If $x\not \in m$ then $x$ must be a unit by hypothesis. But that is not possible as $I$ is not trivial and hence $I \subset m$. Thus, $(A, m)$ is a local ring.
\end{proof}

\begin{lemma}
    Let $A$ be a ring, $m$ be a maximal ideal. If every element of $1+m$ is a unit then $(A,m)$ is local.
    \label{lemma:sufficient-local-2}
\end{lemma}
\begin{proof}
    By lemma \ref{lemma:sufficient-local}, it suffices to show that every element of $A\setminus m$ is a unit. So let $x\in A \setminus m$. Then $\left( x \right) + m = A $as $m$ is a maximal ideal. So, there are elements $y\in A$ and $z\in m$ such that $1=xy+z$. Then $xy = 1-z \in 1+m$ and hence $xy$ is a unit. Since $xy$ is a unit, there is some $u \in A$ such that $(xy)u= u(xy)=1$. But by associativity and commutativity, we have that $x(yu)=(yu)x=1$ and hence $x$ is a unit.
\end{proof} 

\subsection{Introduction to Modules}

\begin{definition}
    Let $A$ be a ring. An $A$-module is a abelian group $M$ with a multiplication map
    $$\cdot \, : A\times M \to M $$
    $$(a \cdot x ) \mapsto ax$$
    satisfying
    \begin{enumerate}[label=(\roman*)]
	\item $a(x+y)=ax+ay$ for all $a\in A$ and $x,y \in M$,
	\item $\left( a+b \right) x = ax+bx$ for all $a,b\in A$ and $x\in M$,
	\item $(ab)x=a(bx)$ for all $a,b\in A$ and $x\in M$,
	\item $1_A x = x$ for $x\in M$.
    \end{enumerate}
    \label{def:module}

\end{definition}

Alternatively, an $A$-module is an abelian group $M$ together with a ring homomorphism $\varphi : A \to \text{End} (M)$ where $\text{End} (M)$ is the ring of endomorphism of the abelian group $M$. Recall that sum in the ring $\text{End} (M)$ is given pointwise and the multiplication is given by function composition. 

To check the equivalence of two definitions, let $M$ be a $A$-module in the sense of Definition \ref{def:module}. Define a map $\varphi : A \to \text{End} (M)$ by $a \stackrel{\varphi}{\mapsto} \varphi _a$ where $\varphi_a : M \to M$ given by $\varphi _a (m ) = am$ for every $m \in M$. It is now easily seen that $\varphi$ is a ring homomorphism. Conversely, let $M$ be a module in the sense of previous paragraph. Now, define $\cdot : A \to M \times M$ by $(a \cdot m ) = (\varphi (a)) (m)$. It is easy to check the properties (i)--(iv) of Definition \ref{def:module}.

\begin{definition}
    A $A$-module $M$ is said to be \textit{faithful} if the map $\varphi : A \to \text{End} (M)$ is injective.
    \label{def:module-faithful}
\end{definition}

\begin{example}
Here are a few examples of modules:
    \begin{enumerate} 	\item Every vector space over a field $k$ is a $k$-module.
	\item Every abelian group is a $\Z$-module.
    \end{enumerate}
\end{example}

